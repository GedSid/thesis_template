%%%%%%%%%%%%%%%%%%%%%%%%%%%%%%%%%%%%%%%%%%%%%%%%%%%%%%%%%%%%%%%%%%%%%%%%%%%%%%%%
%
% Template license:
% CC BY-NC-SA 3.0 (http://creativecommons.org/licenses/by-nc-sa/3.0/)
%
%%%%%%%%%%%%%%%%%%%%%%%%%%%%%%%%%%%%%%%%%%%%%%%%%%%%%%%%%%%%%%%%%%%%%%%%%%%%%%%%

%   PACKAGES AND OTHER DOCUMENT CONFIGURATIONS

\documentclass[
11pt, % The default document font size, options: 10pt, 11pt, 12pt
%oneside, % Two side (alternating margins) for binding by default, uncomment to switch to one side
%chapterinoneline,% Have the chapter title next to the number in one single line
spanish,
singlespacing, % Single line spacing, alternatives: onehalfspacing or doublespacing
%draft, % Uncomment to enable draft mode (no pictures, no links, overfull hboxes indicated)
%nolistspacing, % If the document is onehalfspacing or doublespacing, uncomment this to set spacing in lists to single
%liststotoc, % Uncomment to add the list of figures/tables/etc to the table of contents
%toctotoc, % Uncomment to add the main table of contents to the table of contents
parskip, % Uncomment to add space between paragraphs
%codirector, % Uncomment to add a codirector to the title page
headsepline, % Uncomment to get a line under the header
]{MastersDoctoralThesis} % The class file specifying the document structure

%----------------------------------------------------------------------------------------
%	INFORMACIÓN DE LA MEMORIA
%----------------------------------------------------------------------------------------

\thesistitle{Implementación de Interfaz SDI en FPGA} % El títulos de la memoria, se usa en la carátula y se puede usar el cualquier lugar del documento con el comando \ttitle

% Nombre del posgrado, se usa en la carátula y se puede usar el cualquier lugar del documento con el comando \degreename
\posgrado{Carrera de Especialización en Sistemas Embebidos}

\author{Ing. Joaquin Gaspar Ulloa} % Tu nombre, se usa en la carátula y se puede usar el cualquier lugar del documento con el comando \authorname

\director{Mg. Ing. XXXXXX (FIUBA)} % El nombre del director, se usa en la carátula y se puede usar el cualquier lugar del documento con el comando \dirname
%\codirector{Nombre del codirector (pertenencia)} % El nombre del codirector si lo hubiera, se usa en la carátula y se puede usar el cualquier lugar del documento con el comando \codirname.  Para activar este campo se debe descomentar la opción "codirector" en el comando \documentclass, línea 23.

\juradoUNO{Dr. Ing. XXXXXX (UTN)} % Nombre y pertenencia del un jurado se usa en la carátula y se puede usar el cualquier lugar del documento con el comando \jur1name
\juradoDOS{Mg. Ing. XXXXXX (FIUBA)} % Nombre y pertenencia del un jurado se usa en la carátula y se puede usar el cualquier lugar del documento con el comando \jur2name
\juradoTRES{Esp. Ing. XXXXXX (FIUBA)} % Nombre y pertenencia del un jurado se usa en la carátula y se puede usar el cualquier lugar del documento con el comando \jur3name

\ciudad{Ciudad Autónoma de Buenos Aires}

\fechaINICIO{octubre de 2020}
\fechaFINAL{junio de 2024}

\keywords{Sistemas Embebidos, FPGA, Video Digital, FIUBA} % Keywords for your thesis, print it elsewhere with \keywordnames

\begin{document}

\frontmatter % Use roman page numbering style (i, ii, iii, iv...) for the pre-content pages

\pagestyle{plain} % Default to the plain heading style until the thesis style is called for the body content

%	RESUMEN - ABSTRACT 

\begin{abstract}
\addchaptertocentry{\abstractname} % Add the abstract to the table of contents

%The Thesis Abstract is written here (and usually kept to just this page). The page is kept centered vertically so can expand into the blank space above the title too\ldots
\centering

TBD

\end{abstract}

%	CONTENIDO DE LA MEMORIA  - AGRADECIMIENTOS

\begin{acknowledgements}
%\addchaptertocentry{\acknowledgementname} % Descomentando esta línea se puede agregar los agradecimientos al índice
\vspace{1.5cm}

Quisiera agradecer al tutor del trabajo, mis compañeros de la carrera y familiares que permitieron completar este documento.

\end{acknowledgements}

\tableofcontents % Prints the main table of contents

\listoffigures % Prints the list of figures

\listoftables % Prints the list of tables

%\dedicatory{\textbf{Dedicado a... [OPCIONAL]}}  % escribir acá si se desea una dedicatoria

%   CONTENIDO DE LA MEMORIA  - CAPÍTULOS

\mainmatter % Begin numeric (1,2,3...) page numbering

\pagestyle{thesis} % Return the page headers back to the "thesis" style

\chapter{Introducción General}\label{Chapter1}

El capítulo presenta los temas principales abordados en el trabajo con la
intención de orientar al lector en las estrategias de desarrollo
implementadas.

\section{Estado del Arte}

La Tabla~\ref{tab:sdi_standards} muestra la evolución de los estándares a lo largo del tiempo:

\begin{table}[h]
    \centering
    \begin{tabular}{cccccc}
        \toprule
        \textbf{Standard} & \textbf{Nombre} & \textbf{Fecha} & \textbf{Bitrates} & \textbf{Ejemplos de aplicación} \\
        \midrule
        SMPTE 259M      & SD-SDI    & 1989 & 360 Mbit/s     & 480i, 576i \\
        SMPTE 344M      & ED-SDI    & 2000 & 540 Mbit/s     & 480p, 576p \\
        SMPTE 292M      & HD-SDI    & 1998 & 1.485 Gbit/s   & 720p, 1080i \\
        SMPTE 372M      & Dual Link & 2002 & 2.970 Gbit/s   & 1080p60 \\
        SMPTE 424M      & 3G-SDI    & 2006 & 2.970 Gbit/s   & 1080p60 \\
        SMPTE ST-2081   & 6G-SDI    & 2015 & 6 Gbit/s       & 2160p30 \\
        SMPTE ST-2082   & 12G-SDI   & 2015 & 12 Gbit/s      & 2160p60 \\
        SMPTE ST-2083   & 24G-SDI   & 2020 & 24 Gbit/s      & 2160p/4k@120, 8k@60 \\
        \bottomrule
    \end{tabular}
    \caption{SDI Standards and Characteristics}\label{tab:sdi_standards}
\end{table}

\section{Motivación}

\vspace{1cm}
\begin{figure}[htbp]
    \centering
    \begin{minipage}{.45\linewidth}
        \includegraphics[width=\linewidth]{./Figures/DMUX-OEM.jpg}
        % \caption{First caption}
        % \label{fig:vs-mux}
    \end{minipage}
    \hspace{.05\linewidth}
    \begin{minipage}{.45\linewidth}
        \includegraphics[width=\linewidth]{./Figures/ECD-3000.png}
        % \caption{Second caption}
        % \label{fig:vs-ecd}
    \end{minipage}
    \caption{Productos VideoSwitch, Izquierda: Multiplexor Digital - Derecha: Encoder H.264}
        \label{fig:vs-mux-ecd}
\end{figure}
\vspace{1cm}

VideoSwitch SRL \citep{vs-srl} es la empresa líder en Argentina y Hispanoamérica
en el desarrollo de equipos, soluciones y servicios para la transmisión y
acondicionamiento de video en la industria de la televisión.


%   CONTENIDO DE LA MEMORIA  - APÉNDICES

% \appendix % indicativo para indicarle a LaTeX los siguientes "capítulos" son apéndices

% Incluir los apéndices de la memoria como archivos separadas desde la carpeta Appendices
% Descomentar las líneas a medida que se escriben los apéndices

% \include{Appendices/AppendixA}
% \include{Appendices/AppendixB}

%	BIBLIOGRAPHY

\Urlmuskip=0mu plus 1mu\relax
\raggedright
\printbibliography[heading=bibintoc]

\end{document}  
